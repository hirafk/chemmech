\documentclass[a4paper]{article}
\usepackage{luatexja}            % Japanese
\usepackage{pgfplots}
\usepackage{graphics}
\usepackage{chemfig}
\usepackage{mol2chemfig}     % mol2chemfigパッケージ
\usepackage{tikz}
\usepackage{xstring}
\usepackage{chemmech}
\pgfplotsset{compat=1.18}
\usetikzlibrary{calc} 
\title{Macros for electron-pushing arrows}
\author{Name}
\date{\today}
\begin{document}
\maketitle
ChemMech Package Example Automated electron pushing arrows for \\
ChemFig
\section*{Bond Modifications}
\vspace{1cm}
\textbf{cleavage}
\schemestart
\chemfig{
                     @{a1}O%
    =[@{a1-2}:210]@{a2}%
                        (
        -[@{a2-3}:150]@{a3}%
                        )
    -[@{a2-4}:270]@{a4}%
}
\bdclev[1]{a2-a1}
\schemestop
\hspace{2cm}
\textbf{form}
\schemestart
\chemfig{
                     @{a1}\charge{45 = $\scriptstyle-$}{O}%
    -[@{a1-2}:210]@{a2}\charge{-30 = $\scriptstyle+$}{}%
                        (
        -[@{a2-3}:150]@{a3}%
                        )
    -[@{a2-4}:270]@{a4}%
}
\bdform[1]{a2-a1}
\schemestop
\\
\vspace{2cm}
\textbf{relay}
\schemestart
\chemfig{
                  @{a3}%
    =[@{a1-3}:330]@{a1}%
     -[@{a1-2}:30]@{a2}%
    =[@{a2-4}:330]@{a4}%
}
\bdrelay{a3-1-2}
\bdrelay{a1-2-4}
\schemestop
\section*{Intermolecular Attacks}
\vspace{1cm}
\textbf{nucleophilic attack}\qquad
\schemestart
\chemfig{@{n}\charge{45 = $\scriptstyle-$}{Nu}}
\qquad
    \chemfig{
                  @{c2}%
    -[@{c1-2}:210]@{c1}\charge{-45 = $\scriptstyle+$}{}%
                     (
        -[@{c1-3}:150]@{c3}%
                     )
    -[@{c1-4}:270]@{c4}%
    }
\nuatk[90]{n}{c1-c4}
\schemestop
\\

\vspace{2cm}
\textbf{electrophilic attack}\qquad
\schemestart
\chemfig{
                   @{c1}%
    =[@{c1-2}:330]@{c2}%
     -[@{c2-4}:270]@{c4}%
    =[@{c4-6}:210]@{c6}%
     -[@{c5-6}:150]@{c5}%
     =[@{c3-5}:90]@{c3}%
                      (
          -[@{c1-3}:30]% -> 1
                      )
}
\qquad
\chemfig{@{e}\charge{45 = $\scriptstyle+$}{E}}
\elatk[90]{c2-1}{e}
\schemestop
\\
\subsection*{EX. E2 elimination}\qquad
\chemfig{@{n}\charge{45 = $\scriptstyle-$}{B}}
\qquad
\chemfig{
                        @{c1}Br%
    -[@{c1-2}:210,1.613]@{c2}C%
                           (
              -[@{c2-4}:230]@{c4}H%
                           )
                           (
              -[@{c2-5}:310]@{c5}H%
                           )
    -[@{c2-3}:150,1.613]@{c3}C%
                           (
               -[@{c3-6}:60]@{c6}H%
                           )
                           (
              -[@{c3-7}:150]@{c7}H%
                           )
          -[@{c3-8}:240]@{c8}H%
}
\bdclev[1]{c2-c1}
\bdrelay{c8-c3-c2}
\nuatk[270]{n}{c8-c3}
\subsection*{EX. hemiacetal reaction}
\schemestart
\chemfig{
                      H@{b1}O%
    -[@{b1-2}:0,,2,1]@{b2}CH_3%
}
\qquad
\chemfig{
                  @{c1}O%
    =[@{c1-2}:270,0.7]@{c2}%
                     (
        -[@{c2-3}:210]@{c3}%
                     )
    -[@{c2-4}:330]@{c4}%
}
\nuatk[270]{b1}{c2-1}
\bdclev[1]{c2-c1}
\arrow
\chemfig{
                    @{a2}HO%
    -[@{a2-3}:30,,2]@{a3}%
                       (
          -[@{a3-4}:120]@{a4}%
                       )
                       (
          -[@{a3-5}:300]@{a5}%
                       )
       -[@{a1-3}:30]@{a1}O%
        (-[:60]@{a7}H)
      -[@{a1-6}:330]@{a6}%%
}
\schemestop
\end{document}
%================ヘミセアセタール===================